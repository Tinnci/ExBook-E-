% 张宇预测卷 - 第1套 错题本
% 使用 ExBook 文档类,A4标准版本
\documentclass[standard]{ExBook} 

% 在 document 之前设置主题颜色
\setThemeColor{\blue}

% 加载解答显示配置(可在 solution_config.tex 中切换显示/隐藏)
% Hide solutions version (for self-testing)
\setSolutionDisplay{\hideSolution}


\begin{document}

% 封面设置
\CoverImg{img/cover.jpg}
\PreTitle{张宇预测卷}
\Title{第1套·填空选择题}
\TitleDescription{考研数学错题本}
\Creator{学生}
\UpdateTime{2025年10月28日}

% 页眉页脚设置
\Lhead{张宇预测卷}
\Chead{第1套}
\Rhead{填空选择题}

% 加载封面
\maketitle 

\setcounter{page}{1}
\tableofcontents 
    
\clearpage 

\section{张宇预测卷·第1套}
\subsection{填空题和选择题}

\begin{qitems}

    \begin{bbox}
        \qitem 设总体$X\sim N(\mu,1)$,$H_0:\mu=0, H_1:\mu=1$. 来自总体$X$的样本容量为$9$的简单随机样本均值为$\bar{X}$,设拒绝域为$W=\{\bar{X}\ge 0.55\}$,则不犯第二类错误的概率为
        \begin{itemize}
            \item[A.] $1-\Phi(1.35)$
            \item[B.] $\Phi(1.35)$
            \item[C.] $\Phi(1.65)$
            \item[D.] $1-\Phi(1.65)$
        \end{itemize}
        \begin{solution}
            \textbf{解题步骤}
            
            \textbf{1. 理解第二类错误及其概率}
            \begin{itemize}
                \item 第二类错误(Type II Error)是指原假设$H_0$不成立,但我们没有拒绝$H_0$(即接受了$H_0$)。
                \item 犯第二类错误的概率通常记为$\beta$。
                \item $\beta = P(\text{接受 } H_0 | H_1 \text{ 为真})$。
                \item 本题要求的是"不犯第二类错误的概率",这个概率就是统计检验中的\textbf{功效(Power)},等于$1-\beta$。
                \item 功效的定义是:当备择假设$H_1$为真时,我们能够正确地拒绝原假设$H_0$的概率。即$1-\beta = P(\text{拒绝 } H_0 | H_1 \text{ 为真})$。
            \end{itemize}
            
            \textbf{2. 确定检验的条件}
            \begin{itemize}
                \item 拒绝域为$W=\{\bar{X}\ge 0.55\}$。
                \item 备择假设$H_1$为真,意味着总体的真实均值为$\mu=1$。
                \item 总体方差$\sigma^2=1$,样本容量$n=9$。
                \item 根据中心极限定理,样本均值$\bar{X}$的分布为$\bar{X}\sim N(\mu, \frac{\sigma^2}{n})$。
                \item 当$H_1$为真时,$\mu=1$,所以$\bar{X}\sim N(1, \frac{1}{9})$。
            \end{itemize}
            
            \textbf{3. 计算不犯第二类错误的概率}
            \begin{itemize}
                \item 我们需要计算$P(\bar{X}\in W | \mu=1)$,即$P(\bar{X}\ge 0.55 | \mu=1)$。
                \item 标准化公式为$Z=\frac{\bar{X}-\mu}{\sigma/\sqrt{n}}$。
                \item 在这里,$\mu=1$, $\sigma=1$, $n=9$,所以标准差为$\frac{\sigma}{\sqrt{n}}=\frac{1}{\sqrt{9}}=\frac{1}{3}$。
                \item $P(\bar{X}\ge 0.55) = P\left(\frac{\bar{X}-1}{1/3} \ge \frac{0.55-1}{1/3}\right) = P(Z \ge -1.35)$
                \item 根据标准正态分布的对称性,$P(Z \ge -z) = P(Z \le z)$。
                \item 所以,$P(Z \ge -1.35) = P(Z \le 1.35) = \Phi(1.35)$。
            \end{itemize}
            
            \textbf{最终答案}:$B$($\Phi(1.35)$)
        \end{solution}
    \end{bbox}

    \begin{bbox}
        \qitem $z=\arcsin y^x$在点$(-1,2)$处的全微分为$dz=\blankline.$
        \begin{solution}
            \textbf{解题步骤}
            
            \textbf{1. 全微分公式}
            
            函数$z=f(x,y)$的全微分公式为:$dz = \frac{\partial z}{\partial x}dx + \frac{\partial z}{\partial y}dy$。我们需要先求出$z$对$x$和$y$的偏导数。
            
            \textbf{2. 求偏导数$\frac{\partial z}{\partial x}$}
            
            将$y$视为常数,对$x$求导。根据链式法则和基本求导公式$(\arcsin u)'=\frac{1}{\sqrt{1-u^2}}$和$(a^x)'=a^x\ln a$:
            $$\frac{\partial z}{\partial x} = \frac{1}{\sqrt{1-(y^x)^2}} \cdot \frac{\partial(y^x)}{\partial x} = \frac{y^x\ln y}{\sqrt{1-y^{2x}}}$$
            
            \textbf{3. 求偏导数$\frac{\partial z}{\partial y}$}
            
            将$x$视为常数,对$y$求导。根据链式法则和基本求导公式$(\arcsin u)'=\frac{1}{\sqrt{1-u^2}}$和$(y^n)'=ny^{n-1}$:
            $$\frac{\partial z}{\partial y} = \frac{1}{\sqrt{1-(y^x)^2}} \cdot \frac{\partial(y^x)}{\partial y} = \frac{xy^{x-1}}{\sqrt{1-y^{2x}}}$$
            
            \textbf{4. 计算在点$(-1,2)$处的偏导数值}
            
            将$x=-1, y=2$代入上述偏导数表达式:
            \begin{itemize}
                \item $\frac{\partial z}{\partial x}|_{(-1,2)} = \frac{2^{-1}\ln 2}{\sqrt{1-2^{-2}}} = \frac{\frac{1}{2}\ln 2}{\sqrt{1-\frac{1}{4}}} = \frac{\frac{1}{2}\ln 2}{\frac{\sqrt{3}}{2}} = \frac{\ln 2}{\sqrt{3}} = \frac{\sqrt{3}}{3}\ln 2$
                \item $\frac{\partial z}{\partial y}|_{(-1,2)} = \frac{(-1) \cdot 2^{-2}}{\sqrt{1-\frac{1}{4}}} = \frac{-\frac{1}{4}}{\frac{\sqrt{3}}{2}} = -\frac{1}{2\sqrt{3}} = -\frac{\sqrt{3}}{6}$
            \end{itemize}
            
            \textbf{5. 写出全微分表达式}
            
            将计算出的偏导数值代入全微分公式。
            
            \textbf{最终答案}:$dz = \frac{\sqrt{3}}{3}\ln 2\, dx - \frac{\sqrt{3}}{6}\, dy$
        \end{solution}
    \end{bbox}

    \begin{bbox}
        \qitem 设$e^{ax}\ge 1+x$对任意实数$x$均成立,则$a$的取值范围为$\blankline.$
        \begin{solution}
            \textbf{解题步骤}
            
            \textbf{1. 构造辅助函数}
            
            设函数$f(x) = e^{ax} - 1 - x$。题目条件等价于$f(x)\ge 0$对任意实数$x$恒成立。这意味着函数$f(x)$的全局最小值必须大于或等于$0$。
            
            \textbf{2. 求函数的最小值}
            
            对$f(x)$求导以寻找极值点:
            $$f'(x) = ae^{ax} - 1$$
            
            令$f'(x)=0$,得到$ae^{ax}=1$,即$e^{ax}=\frac{1}{a}$。
            \begin{itemize}
                \item 要使该方程有解,必须有$\frac{1}{a}>0$,即$a>0$。
                \item 如果$a=0$,不等式为$1\ge 1+x$,化为$x\le 0$,不满足对任意$x$成立。
                \item 如果$a<0$,则$e^{ax}>0$而$\frac{1}{a}<0$,方程无解。此时$f'(x)=ae^{ax}-1$恒小于0,函数单调递减,不可能恒大于等于0。
                \item 因此,必须有$a>0$。
            \end{itemize}
            
            \textbf{3. 确定极值点和最小值}
            
            当$a>0$时,解$e^{ax}=\frac{1}{a}$得$x_0 = -\frac{\ln a}{a}$是唯一的驻点。
            
            求二阶导数判断极值类型:$f''(x) = a^2 e^{ax} > 0$恒成立,所以$x_0$是全局最小点。
            
            \textbf{4. 建立关于$a$的不等式}
            
            函数$f(x)$的最小值为:
            $$f(x_0) = e^{-\ln a} - 1 + \frac{\ln a}{a} = \frac{1}{a} - 1 + \frac{\ln a}{a} \ge 0$$
            
            化简得:$1 - a + \ln a \ge 0$,即$\ln a \ge a - 1$。
            
            \textbf{5. 解关于$a$的不等式}
            
            分析函数$g(a) = \ln a - (a-1)$在$a>0$时的性质。
            $$g'(a) = \frac{1}{a} - 1$$
            
            令$g'(a)=0$,解得$a=1$。当$0<a<1$时,$g'(a)>0$;当$a>1$时,$g'(a)<0$。因此$a=1$是最大值点。
            
            $g(a)$的最大值为$g(1) = 0$。因为$g(a)$的最大值是$0$,所以$g(a)\ge 0$当且仅当$a=1$。
            
            \textbf{最终答案}:$a=1$
        \end{solution}
    \end{bbox}

    \begin{bbox}
        \qitem 已知$\Omega=\{(x,y,z)|y^2+z^2\le 1, 0\le x\le 1\}$,$\Sigma$为$\Omega$的边界面且取外侧,则$\oiint_{\Sigma} (y^3+z\sin x)dydz + zdxdy=\blankline.$
        \begin{solution}
            \textbf{解题步骤}
            
            \textbf{1. 应用高斯散度定理}
            
            该积分是第二类曲面积分,区域$\Omega$是封闭的,曲面$\Sigma$取外侧,满足高斯公式的应用条件。
            
            \textbf{2. 确定$P, Q, R$并计算散度}
            
            从积分表达式$\oiint_{\Sigma} Pdydz + Qdzdx + Rdxdy$中:
            \begin{itemize}
                \item $P = y^3+z\sin x$
                \item $Q = 0$
                \item $R = z$
            \end{itemize}
            
            计算散度:
            $$\nabla \cdot \mathbf{F} = \frac{\partial P}{\partial x} + \frac{\partial Q}{\partial y} + \frac{\partial R}{\partial z} = 0 + 3y^2 + 1 = 3y^2+1$$
            
            \textbf{3. 转化为三重积分}
            
            由高斯公式:
            $$\oiint_{\Sigma} Pdydz + Qdzdx + Rdxdy = \iiint_{\Omega} (3y^2+1) dV$$
            
            \textbf{4. 计算三重积分}
            
            先对$yz$平面上的圆盘$D: y^2+z^2\le 1$积分,再对$x$积分。使用极坐标变换:$y=r\cos\theta, z=r\sin\theta$。
            
            $$\iint_{D} (3y^2+1) dydz = \int_0^{2\pi} \int_0^1 (3r^2\cos^2\theta+1) r\, dr\, d\theta$$
            
            先对$r$积分:
            $$\int_0^1 (3r^3\cos^2\theta+r) dr = \frac{3}{4}\cos^2\theta + \frac{1}{2}$$
            
            再对$\theta$积分,利用$\cos^2\theta = \frac{1+\cos(2\theta)}{2}$:
            $$\int_0^{2\pi} \left(\frac{3}{4}\cos^2\theta + \frac{1}{2}\right) d\theta = \frac{7\pi}{4}$$
            
            完成对$x$的积分:
            $$\iiint_{\Omega} (3y^2+1) dV = \int_0^1 \frac{7\pi}{4} dx = \frac{7\pi}{4}$$
            
            \textbf{最终答案}:$\frac{7\pi}{4}$
        \end{solution}
    \end{bbox}

    \begin{bbox}
        \qitem 设随机变量$X\sim B(2, \frac{1}{2})$,则$E(e^{2X})=\blankline.$
        \begin{solution}
            \textbf{解题步骤}
            
            \textbf{方法一:利用矩母函数(MGF)}
            
            \begin{itemize}
                \item 随机变量$X$的矩母函数定义为$M_X(t)=E(e^{tX})$。
                \item 对于服从二项分布$B(n,p)$的随机变量,其矩母函数为$M_X(t)=(1-p+pe^t)^n$。
                \item 本题中,$n=2, p=\frac{1}{2}$,所以$X$的矩母函数为:
                    $$M_X(t) = (1-\frac{1}{2}+\frac{1}{2}e^t)^2 = \left(\frac{1+e^t}{2}\right)^2$$
                \item 题目所求为$E(e^{2X})$,这正好是矩母函数在$t=2$处的值。
                \item $$E(e^{2X}) = M_X(2) = \left(\frac{1+e^2}{2}\right)^2 = \frac{(1+e^2)^2}{4}$$
            \end{itemize}
            
            \textbf{方法二:利用期望的定义}
            
            \begin{itemize}
                \item $X\sim B(2, \frac{1}{2})$,所以$X$可能的取值为$0, 1, 2$。
                \item 其概率分布:$P(X=0) = \dfrac{1}{4}$,$P(X=1) = \dfrac{1}{2}$,$P(X=2) = \dfrac{1}{4}$
                \item 根据期望的定义:
                    $$E(e^{2X}) = e^{0} \cdot \frac{1}{4} + e^{2} \cdot \frac{1}{2} + e^{4} \cdot \frac{1}{4} = \frac{1+2e^2+e^4}{4}$$
                \item 分子是完全平方式:$(1+e^2)^2 = 1^2 + 2 \cdot 1 \cdot e^2 + (e^2)^2 = 1+2e^2+e^4$
                \item 所以,$$E(e^{2X}) = \frac{(1+e^2)^2}{4}$$
            \end{itemize}
            
            \textbf{最终答案}:$\dfrac{(1+e^2)^2}{4}$(或$\dfrac{1+2e^2+e^4}{4}$)
        \end{solution}
    \end{bbox}

\end{qitems}

\end{document}
