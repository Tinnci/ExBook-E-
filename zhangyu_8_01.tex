% 张宇预测卷 - 第1套 错题本
% 使用 ExBook 文档类,A4标准版本
\documentclass[standard]{ExBook} 

% 在 document 之前设置主题颜色
\setThemeColor{\blue}

% 加载解答显示配置(可在 solution_config.tex 中切换显示/隐藏)
% Hide solutions version (for self-testing)
\setSolutionDisplay{\hideSolution}


\begin{document}

% 封面设置
\CoverImg{img/cover.jpg}
\PreTitle{张宇预测卷}
\Title{第1套·填空选择题}
\TitleDescription{考研数学错题本}
\Creator{学生}
\UpdateTime{2025年10月28日}

% 页眉页脚设置
\Lhead{张宇预测卷}
\Chead{第1套}
\Rhead{填空选择题}

% 加载封面
\maketitle 

\setcounter{page}{1}
\tableofcontents 

\section{张宇预测卷·第1套}
\subsection{填空题和选择题}

\begin{qitems}

    \begin{bbox}
        \qitem 设总体$X\sim N(\mu,1)$,$H_0:\mu=0, H_1:\mu=1$. 来自总体$X$的样本容量为$9$的简单随机样本均值为$\bar{X}$,设拒绝域为$W=\{\bar{X}\ge 0.55\}$,则不犯第二类错误的概率为
        \begin{itemize}
            \item[A.] $1-\Phi(1.35)$
            \item[B.] $\Phi(1.35)$
            \item[C.] $\Phi(1.65)$
            \item[D.] $1-\Phi(1.65)$
        \end{itemize}
        \begin{solution}
            \textbf{解题步骤}
            
            \textbf{1. 理解第二类错误及其概率}
            \begin{itemize}
                \item 第二类错误(Type II Error)是指原假设$H_0$不成立,但我们没有拒绝$H_0$(即接受了$H_0$)。
                \item 犯第二类错误的概率通常记为$\beta$。
                \item $\beta = P(\text{接受 } H_0 | H_1 \text{ 为真})$。
                \item 本题要求的是"不犯第二类错误的概率",这个概率就是统计检验中的\textbf{功效(Power)},等于$1-\beta$。
                \item 功效的定义是:当备择假设$H_1$为真时,我们能够正确地拒绝原假设$H_0$的概率。即$1-\beta = P(\text{拒绝 } H_0 | H_1 \text{ 为真})$。
            \end{itemize}
            
            \textbf{2. 确定检验的条件}
            \begin{itemize}
                \item 拒绝域为$W=\{\bar{X}\ge 0.55\}$。
                \item 备择假设$H_1$为真,意味着总体的真实均值为$\mu=1$。
                \item 总体方差$\sigma^2=1$,样本容量$n=9$。
                \item 根据中心极限定理,样本均值$\bar{X}$的分布为$\bar{X}\sim N(\mu, \frac{\sigma^2}{n})$。
                \item 当$H_1$为真时,$\mu=1$,所以$\bar{X}\sim N(1, \frac{1}{9})$。
            \end{itemize}
            
            \textbf{3. 计算不犯第二类错误的概率}
            \begin{itemize}
                \item 我们需要计算$P(\bar{X}\in W | \mu=1)$,即$P(\bar{X}\ge 0.55 | \mu=1)$。
                \item 标准化公式为$Z=\frac{\bar{X}-\mu}{\sigma/\sqrt{n}}$。
                \item 在这里,$\mu=1$, $\sigma=1$, $n=9$,所以标准差为$\frac{\sigma}{\sqrt{n}}=\frac{1}{\sqrt{9}}=\frac{1}{3}$。
                \item $P(\bar{X}\ge 0.55) = P\left(\frac{\bar{X}-1}{1/3} \ge \frac{0.55-1}{1/3}\right) = P(Z \ge -1.35)$
                \item 根据标准正态分布的对称性,$P(Z \ge -z) = P(Z \le z)$。
                \item 所以,$P(Z \ge -1.35) = P(Z \le 1.35) = \Phi(1.35)$。
            \end{itemize}
            
            \textbf{最终答案}:$B$($\Phi(1.35)$)
        \end{solution}
    \end{bbox}

    \begin{bbox}
        \qitem $z=\arcsin y^x$在点$(-1,2)$处的全微分为$dz=\blankline.$
        \begin{solution}
            \textbf{解题步骤}
            
            \textbf{1. 全微分公式}
            
            函数$z=f(x,y)$的全微分公式为:$dz = \frac{\partial z}{\partial x}dx + \frac{\partial z}{\partial y}dy$。我们需要先求出$z$对$x$和$y$的偏导数。
            
            \textbf{2. 求偏导数$\frac{\partial z}{\partial x}$}
            
            将$y$视为常数,对$x$求导。根据链式法则和基本求导公式$(\arcsin u)'=\frac{1}{\sqrt{1-u^2}}$和$(a^x)'=a^x\ln a$:
            $$\frac{\partial z}{\partial x} = \frac{1}{\sqrt{1-(y^x)^2}} \cdot \frac{\partial(y^x)}{\partial x} = \frac{y^x\ln y}{\sqrt{1-y^{2x}}}$$
            
            \textbf{3. 求偏导数$\frac{\partial z}{\partial y}$}
            
            将$x$视为常数,对$y$求导。根据链式法则和基本求导公式$(\arcsin u)'=\frac{1}{\sqrt{1-u^2}}$和$(y^n)'=ny^{n-1}$:
            $$\frac{\partial z}{\partial y} = \frac{1}{\sqrt{1-(y^x)^2}} \cdot \frac{\partial(y^x)}{\partial y} = \frac{xy^{x-1}}{\sqrt{1-y^{2x}}}$$
            
            \textbf{4. 计算在点$(-1,2)$处的偏导数值}
            
            将$x=-1, y=2$代入上述偏导数表达式:
            \begin{itemize}
                \item $\frac{\partial z}{\partial x}|_{(-1,2)} = \frac{2^{-1}\ln 2}{\sqrt{1-2^{-2}}} = \frac{\frac{1}{2}\ln 2}{\sqrt{1-\frac{1}{4}}} = \frac{\frac{1}{2}\ln 2}{\frac{\sqrt{3}}{2}} = \frac{\ln 2}{\sqrt{3}} = \frac{\sqrt{3}}{3}\ln 2$
                \item $\frac{\partial z}{\partial y}|_{(-1,2)} = \frac{(-1) \cdot 2^{-2}}{\sqrt{1-\frac{1}{4}}} = \frac{-\frac{1}{4}}{\frac{\sqrt{3}}{2}} = -\frac{1}{2\sqrt{3}} = -\frac{\sqrt{3}}{6}$
            \end{itemize}
            
            \textbf{5. 写出全微分表达式}
            
            将计算出的偏导数值代入全微分公式。
            
            \textbf{最终答案}:$dz = \frac{\sqrt{3}}{3}\ln 2\, dx - \frac{\sqrt{3}}{6}\, dy$
        \end{solution}
    \end{bbox}

    \begin{bbox}
        \qitem 设$e^{ax}\ge 1+x$对任意实数$x$均成立,则$a$的取值范围为$\blankline.$
        \begin{solution}
            \textbf{解题步骤}
            
            \textbf{1. 构造辅助函数}
            
            设函数$f(x) = e^{ax} - 1 - x$。题目条件等价于$f(x)\ge 0$对任意实数$x$恒成立。这意味着函数$f(x)$的全局最小值必须大于或等于$0$。
            
            \textbf{2. 求函数的最小值}
            
            对$f(x)$求导以寻找极值点:
            $$f'(x) = ae^{ax} - 1$$
            
            令$f'(x)=0$,得到$ae^{ax}=1$,即$e^{ax}=\frac{1}{a}$。
            \begin{itemize}
                \item 要使该方程有解,必须有$\frac{1}{a}>0$,即$a>0$。
                \item 如果$a=0$,不等式为$1\ge 1+x$,化为$x\le 0$,不满足对任意$x$成立。
                \item 如果$a<0$,则$e^{ax}>0$而$\frac{1}{a}<0$,方程无解。此时$f'(x)=ae^{ax}-1$恒小于0,函数单调递减,不可能恒大于等于0。
                \item 因此,必须有$a>0$。
            \end{itemize}
            
            \textbf{3. 确定极值点和最小值}
            
            当$a>0$时,解$e^{ax}=\frac{1}{a}$得$x_0 = -\frac{\ln a}{a}$是唯一的驻点。
            
            求二阶导数判断极值类型:$f''(x) = a^2 e^{ax} > 0$恒成立,所以$x_0$是全局最小点。
            
            \textbf{4. 建立关于$a$的不等式}
            
            函数$f(x)$的最小值为:
            $$f(x_0) = e^{-\ln a} - 1 + \frac{\ln a}{a} = \frac{1}{a} - 1 + \frac{\ln a}{a} \ge 0$$
            
            化简得:$1 - a + \ln a \ge 0$,即$\ln a \ge a - 1$。
            
            \textbf{5. 解关于$a$的不等式}
            
            分析函数$g(a) = \ln a - (a-1)$在$a>0$时的性质。
            $$g'(a) = \frac{1}{a} - 1$$
            
            令$g'(a)=0$,解得$a=1$。当$0<a<1$时,$g'(a)>0$;当$a>1$时,$g'(a)<0$。因此$a=1$是最大值点。
            
            $g(a)$的最大值为$g(1) = 0$。因为$g(a)$的最大值是$0$,所以$g(a)\ge 0$当且仅当$a=1$。
            
            \textbf{最终答案}:$a=1$
        \end{solution}
    \end{bbox}

    \begin{bbox}
        \qitem 已知$\Omega=\{(x,y,z)|y^2+z^2\le 1, 0\le x\le 1\}$,$\Sigma$为$\Omega$的边界面且取外侧,则$\oiint_{\Sigma} (y^3+z\sin x)dydz + zdxdy=\blankline.$
        \begin{solution}
            \textbf{解题步骤}
            
            \textbf{1. 应用高斯散度定理}
            
            该积分是第二类曲面积分,区域$\Omega$是封闭的,曲面$\Sigma$取外侧,满足高斯公式的应用条件。
            
            \textbf{2. 确定$P, Q, R$并计算散度}
            
            从积分表达式$\oiint_{\Sigma} Pdydz + Qdzdx + Rdxdy$中:
            \begin{itemize}
                \item $P = y^3+z\sin x$
                \item $Q = 0$
                \item $R = z$
            \end{itemize}
            
            计算散度:
            $$\nabla \cdot \mathbf{F} = \frac{\partial P}{\partial x} + \frac{\partial Q}{\partial y} + \frac{\partial R}{\partial z} = 0 + 3y^2 + 1 = 3y^2+1$$
            
            \textbf{3. 转化为三重积分}
            
            由高斯公式:
            $$\oiint_{\Sigma} Pdydz + Qdzdx + Rdxdy = \iiint_{\Omega} (3y^2+1) dV$$
            
            \textbf{4. 计算三重积分}
            
            先对$yz$平面上的圆盘$D: y^2+z^2\le 1$积分,再对$x$积分。使用极坐标变换:$y=r\cos\theta, z=r\sin\theta$。
            
            $$\iint_{D} (3y^2+1) dydz = \int_0^{2\pi} \int_0^1 (3r^2\cos^2\theta+1) r\, dr\, d\theta$$
            
            先对$r$积分:
            $$\int_0^1 (3r^3\cos^2\theta+r) dr = \frac{3}{4}\cos^2\theta + \frac{1}{2}$$
            
            再对$\theta$积分,利用$\cos^2\theta = \frac{1+\cos(2\theta)}{2}$:
            $$\int_0^{2\pi} \left(\frac{3}{4}\cos^2\theta + \frac{1}{2}\right) d\theta = \frac{7\pi}{4}$$
            
            完成对$x$的积分:
            $$\iiint_{\Omega} (3y^2+1) dV = \int_0^1 \frac{7\pi}{4} dx = \frac{7\pi}{4}$$
            
            \textbf{最终答案}:$\frac{7\pi}{4}$
        \end{solution}
    \end{bbox}

    \begin{bbox}
        \qitem 设随机变量$X\sim B(2, \frac{1}{2})$,则$E(e^{2X})=\blankline.$
        \begin{solution}
            \textbf{解题步骤}
            
            \textbf{方法一:利用矩母函数(MGF)}
            
            \begin{itemize}
                \item 随机变量$X$的矩母函数定义为$M_X(t)=E(e^{tX})$。
                \item 对于服从二项分布$B(n,p)$的随机变量,其矩母函数为$M_X(t)=(1-p+pe^t)^n$。
                \item 本题中,$n=2, p=\frac{1}{2}$,所以$X$的矩母函数为:
                    $$M_X(t) = (1-\frac{1}{2}+\frac{1}{2}e^t)^2 = \left(\frac{1+e^t}{2}\right)^2$$
                \item 题目所求为$E(e^{2X})$,这正好是矩母函数在$t=2$处的值。
                \item $$E(e^{2X}) = M_X(2) = \left(\frac{1+e^2}{2}\right)^2 = \frac{(1+e^2)^2}{4}$$
            \end{itemize}
            
            \textbf{方法二:利用期望的定义}
            
            \begin{itemize}
                \item $X\sim B(2, \frac{1}{2})$,所以$X$可能的取值为$0, 1, 2$。
                \item 其概率分布:$P(X=0) = \dfrac{1}{4}$,$P(X=1) = \dfrac{1}{2}$,$P(X=2) = \dfrac{1}{4}$
                \item 根据期望的定义:
                    $$E(e^{2X}) = e^{0} \cdot \frac{1}{4} + e^{2} \cdot \frac{1}{2} + e^{4} \cdot \frac{1}{4} = \frac{1+2e^2+e^4}{4}$$
                \item 分子是完全平方式:$(1+e^2)^2 = 1^2 + 2 \cdot 1 \cdot e^2 + (e^2)^2 = 1+2e^2+e^4$
                \item 所以,$$E(e^{2X}) = \frac{(1+e^2)^2}{4}$$
            \end{itemize}
            
            \textbf{最终答案}:$\dfrac{(1+e^2)^2}{4}$(或$\dfrac{1+2e^2+e^4}{4}$)
        \end{solution}
    \end{bbox}

    \begin{bbox}
        \qitem 计算二重积分$\int_0^1 dx \int_1^x (e^{-y^2} + e^y \sin y)dy=\blankline.$
        \begin{solution}
            \textbf{解题步骤}
            
            \textbf{1. 分析积分区域}
            
            这道题的关键在于:被积函数中的$e^{-y^2}$不存在初等函数原函数,不能直接对$y$进行积分,因此必须\textbf{交换积分次序}。
            
            原积分为:$I = \int_0^1 dx \int_1^x (e^{-y^2} + e^y \sin y)dy$
            
            观察积分限:当$0 \le x \le 1$时,$1 \le y \le x$。由于在大部分区间内$x < 1$,所以积分上限小于下限,这是"反向"积分。
            
            根据定积分的性质$\int_a^b f(x)dx = -\int_b^a f(x)dx$,将原积分改写为:
            $$I = -\int_0^1 dx \int_x^1 (e^{-y^2} + e^y \sin y)dy$$
            
            现在的积分区域$D$为:$0 \le x \le 1, x \le y \le 1$
            
            这是由直线$x=0$、$y=1$、$y=x$围成的三角形区域,顶点为$(0,0)$、$(0,1)$、$(1,1)$。
            
            \textbf{积分区域详细图示:}
            
            \begin{center}
            \begin{tikzpicture}[scale=3.0, thick]
                % 第一个图:原题的"反向"积分
                \node[font=\Large\bfseries, blue] at (0.5, 2.2) {步骤1:原题中的反向积分};
                \node[font=\small] at (0.5, 2.0) {$I = \int_0^1 dx \int_1^x (\dots)dy$,其中$1 \le y \le x$(反向)};
                
                \begin{scope}[shift={(0,0)}]
                    % 绘制坐标轴
                    \draw[->] (-0.2, -0.2) -- (1.3, -0.2) node[right] {$x$};
                    \draw[->] (-0.2, -0.2) -- (-0.2, 1.3) node[above] {$y$};
                    
                    % 绘制网格(淡化)
                    \draw[gray, very thin, opacity=0.2] (0,0) grid (1.2, 1.2);
                    
                    % 绘制三角形区域(原积分区域 - 用警告颜色表示"反向")
                    \fill[red!10, opacity=0.5] (0,0) -- (0,1) -- (1,1) -- cycle;
                    
                    % 绘制边界线
                    \draw[red, thick, dashed] (0,0) -- (0,1) node[midway, left, font=\small] {$x=0$};
                    \draw[red, thick, dashed] (0,1) -- (1,1) node[midway, above, font=\small] {$y=1$};
                    \draw[red, thick, dashed] (1,1) -- (0,0) node[midway, below right, font=\small] {$y=x$};
                    
                    % 标注顶点
                    \node[below left, font=\tiny] at (0,0) {$(0,0)$};
                    \node[above left, font=\tiny] at (0,1) {$(0,1)$};
                    \node[above right, font=\tiny] at (1,1) {$(1,1)$};
                    
                    % 添加反向箭头(表示y反向积分)
                    \draw[red, ->, very thick, opacity=0.8] (0.3, 0.8) -- (0.3, 0.2);
                    \node[right, red, font=\small\bfseries] at (0.35, 0.5) {$y$反向:$1\to x$};
                    
                    % 标注原点
                    \draw[black, fill] (0,0) circle (1pt);
                \end{scope}
                
                % 箭头指向"转化"
                \node[font=\normalsize, color=purple] at (0.5, -0.5) {使用$\int_a^b = -\int_b^a$};
                \node[font=\Large, color=purple] at (0.5, -1.0) {$\Downarrow$};
                
                % 第二个图:转化后的"正向"积分
                \node[font=\Large\bfseries, blue] at (0.5, -1.5) {步骤2:转化后的正向积分};
                \node[font=\small] at (0.5, -1.7) {$I = -\int_0^1 dx \int_x^1 (\dots)dy$,其中$x \le y \le 1$(正向)};
                
                \begin{scope}[shift={(0,-3.2)}]
                    % 绘制坐标轴
                    \draw[->] (-0.2, -0.2) -- (1.3, -0.2) node[right] {$x$};
                    \draw[->] (-0.2, -0.2) -- (-0.2, 1.3) node[above] {$y$};
                    
                    % 绘制网格(淡化)
                    \draw[gray, very thin, opacity=0.2] (0,0) grid (1.2, 1.2);
                    
                    % 绘制三角形区域(转化后的正向区域)
                    \fill[blue!15, opacity=0.6] (0,0) -- (0,1) -- (1,1) -- cycle;
                    
                    % 绘制边界线
                    \draw[blue, thick, solid] (0,0) -- (0,1) node[midway, left, font=\small] {$x=0$};
                    \draw[blue, thick, solid] (0,1) -- (1,1) node[midway, above, font=\small] {$y=1$};
                    \draw[blue, thick, solid] (1,1) -- (0,0) node[midway, below right, font=\small] {$y=x$};
                    
                    % 标注顶点
                    \node[below left, font=\tiny] at (0,0) {$(0,0)$};
                    \node[above left, font=\tiny] at (0,1) {$(0,1)$};
                    \node[above right, font=\tiny] at (1,1) {$(1,1)$};
                    
                    % 添加箭头表示两种积分方向
                    % 先y后x(红色竖箭头)
                    \draw[red, ->, very thick, opacity=0.7] (0.5, 0.3) -- (0.5, 0.8);
                    \node[right, red, font=\small] at (0.52, 0.55) {先$y$后$x$};
                    
                    % 先x后y(绿色横箭头)
                    \draw[green!60!black, ->, very thick, opacity=0.7] (0.15, 0.6) -- (0.65, 0.6);
                    \node[below, green!60!black, font=\small] at (0.4, 0.58) {先$x$后$y$};
                    
                    % 标注原点
                    \draw[black, fill] (0,0) circle (1pt);
                \end{scope}
            \end{tikzpicture}
            \end{center}
            
            \textbf{2. 交换积分次序}
            
            观察三角形区域:
            \begin{itemize}
                \item $y$的取值范围:$0 \le y \le 1$
                \item 对于固定的$y$,$x$的范围:$0 \le x \le y$(从左边界$x=0$到斜边$x=y$)
            \end{itemize}
            
            交换积分次序后:
            $$I = -\int_0^1 dy \int_0^y (e^{-y^2} + e^y \sin y)dx$$
            
            \textbf{3. 计算新的积分}
            
            \textbf{第一步:计算内层对$x$的积分}
            
            被积函数对$x$积分时可视为常数:
            $$\int_0^y (e^{-y^2} + e^y \sin y)dx = (e^{-y^2} + e^y \sin y) \cdot y = ye^{-y^2} + ye^y\sin y$$
            
            \textbf{第二步:计算外层对$y$的积分}
            
            $$I = -\int_0^1 (ye^{-y^2} + ye^y\sin y)dy = -\left[\int_0^1 ye^{-y^2}dy + \int_0^1 ye^y\sin y\, dy\right]$$
            
            \textbf{计算积分A:$\int_0^1 ye^{-y^2}dy$}
            
            令$u = -y^2$,则$du = -2y\,dy$,故$y\,dy = -\dfrac{1}{2}du$。
            
            当$y=0$时,$u=0$;当$y=1$时,$u=-1$。
            \begin{align*}
            \int_0^1 ye^{-y^2}dy &= \int_0^{-1} e^u \left(-\frac{1}{2}\right)du = \frac{1}{2}\int_{-1}^0 e^u\, du \\
            &= \frac{1}{2}[e^u]_{-1}^0 = \frac{1}{2}(1 - e^{-1}) = \frac{1}{2}\left(1 - \frac{1}{e}\right)
            \end{align*}
            
            \textbf{计算积分B:$\int_0^1 ye^y\sin y\, dy$}
            
            先计算$\int e^y\sin y\, dy$(分部积分两次):
            \begin{align*}
            \int e^y\sin y\, dy &= \frac{1}{2}e^y(\sin y - \cos y) + C
            \end{align*}
            
            对$\int ye^y\sin y\, dy$用分部积分:令$u=y$,$dv=e^y\sin y\, dy$
            $$v = \frac{1}{2}e^y(\sin y - \cos y)$$
            
            \begin{align*}
            \int ye^y\sin y\, dy &= \frac{1}{2}ye^y(\sin y - \cos y) - \frac{1}{2}\int e^y(\sin y - \cos y)dy
            \end{align*}
            
            其中$\int e^y\sin y\, dy = \dfrac{1}{2}e^y(\sin y - \cos y)$,$\int e^y\cos y\, dy = \dfrac{1}{2}e^y(\sin y + \cos y)$
            
            代入计算得:
            $$\int ye^y\sin y\, dy = \frac{1}{2}ye^y(\sin y - \cos y) + \frac{1}{2}e^y\cos y$$
            
            计算定积分:
            \begin{align*}
            \int_0^1 ye^y\sin y\, dy &= \left[\frac{1}{2}ye^y(\sin y - \cos y) + \frac{1}{2}e^y\cos y\right]_0^1 \\
            &= \left[\frac{e}{2}(\sin 1 - \cos 1) + \frac{e}{2}\cos 1\right] - \left[0 + \frac{1}{2}\right] \\
            &= \frac{e}{2}\sin 1 - \frac{1}{2}
            \end{align*}
            
            \textbf{第三步:合并结果}
            
            \begin{align*}
            I &= -\left[\frac{1}{2}\left(1 - \frac{1}{e}\right) + \frac{e}{2}\sin 1 - \frac{1}{2}\right] \\
            &= -\left[\frac{1}{2} - \frac{1}{2e} + \frac{e}{2}\sin 1 - \frac{1}{2}\right] \\
            &= -\left[-\frac{1}{2e} + \frac{e}{2}\sin 1\right] \\
            &= \frac{1}{2e} - \frac{e\sin 1}{2}
            \end{align*}
            
            \textbf{最终答案}:$\dfrac{1}{2e} - \dfrac{e\sin 1}{2}$(或$\dfrac{1}{2e} - \dfrac{e\sin 1}{2}$)
        \end{solution}
    \end{bbox}

    \begin{bbox}
        \qitem 设$y=y(x)$满足$x^2y' + (x^2 - 3)y^2 = 0$且$y(1)=1$。
        
        (1) 求$y=y(x)$的表达式;(2) 计算$\int_0^3 y^2(x)dx$。
        \begin{solution}
            \textbf{解题步骤}
            
            \textbf{(1)求$y=y(x)$的表达式}
            
            \textbf{第一步:分离变量}
            
            原方程为:$x^2y' + (x^2 - 3)y^2 = 0$
            
            整理得:$x^2 y' = -(x^2 - 3)y^2 = (3 - x^2)y^2$
            
            当$y \ne 0$时,两边同时除以$x^2 y^2$并整理:
            $$\frac{dy}{y^2} = \frac{3 - x^2}{x^2} dx$$
            
            \textbf{第二步:两边积分}
            
            左边:$\int y^{-2} dy = -\dfrac{1}{y}$
            
            右边:$\int \dfrac{3 - x^2}{x^2} dx = \int \left(\dfrac{3}{x^2} - 1\right) dx = -\dfrac{3}{x} - x + C$
            
            因此得到通解:
            $$-\frac{1}{y} = -\frac{3}{x} - x + C$$
            
            或写成:
            $$\frac{1}{y} = \frac{3}{x} + x + C_1$$
            (其中$C_1 = -C$为新的常数)
            
            \textbf{第三步:利用初始条件确定常数}
            
            将$y(1)=1$代入:
            $$1 = 3 + 1 + C_1 \implies C_1 = -3$$
            
            \textbf{第四步:得到特解}
            
            代入$C_1 = -3$:
            $$\frac{1}{y} = \frac{3}{x} + x - 3$$
            
            通分:
            $$\frac{1}{y} = \frac{3 + x^2 - 3x}{x} = \frac{x^2 - 3x + 3}{x}$$
            
            因此:
            $$\boxed{y(x) = \frac{x}{x^2 - 3x + 3}}$$
            
            \textbf{(2)计算$\int_0^3 y^2(x)dx$}
            
            \textbf{关键观察:}直接计算$\int_0^3 \dfrac{x^2}{(x^2 - 3x + 3)^2} dx$非常困难。这暗示我们应该进行巧妙的代数分解。
            
            \textbf{第一步:被积函数的分解}
            
            令$D(x) = x^2 - 3x + 3$,$D'(x) = 2x - 3$。
            
            我们尝试将分子$x^2$表示为:
            $$x^2 = A \cdot D(x) + B \cdot D'(x) + C$$
            
            代入:
            $$x^2 = A(x^2 - 3x + 3) + B(2x - 3) + C$$
            
            比较系数:
            \begin{itemize}
                \item $x^2$系数:$A = 1$
                \item $x$系数:$-3A + 2B = 0 \implies B = \dfrac{3}{2}$
                \item 常数项:$3A - 3B + C = 0 \implies C = \dfrac{3}{2}$
            \end{itemize}
            
            因此:
            $$x^2 = (x^2 - 3x + 3) + \frac{3}{2}(2x - 3) + \frac{3}{2}$$
            
            \textbf{第二步:拆分积分}
            
            $$I = \int_0^3 \frac{x^2}{(x^2 - 3x + 3)^2} dx = \int_0^3 \frac{D(x)}{D(x)^2} dx + \frac{3}{2}\int_0^3 \frac{D'(x)}{D(x)^2} dx + \frac{3}{2}\int_0^3 \frac{1}{D(x)^2} dx$$
            
            $$I = I_1 + I_2 + I_3$$
            
            \textbf{计算$I_1 = \int_0^3 \dfrac{1}{D(x)} dx$:}
            
            对$D(x) = x^2 - 3x + 3$配方:
            $$D(x) = \left(x - \frac{3}{2}\right)^2 + \frac{3}{4}$$
            
            使用反正切积分公式$\int \dfrac{1}{u^2 + a^2}du = \dfrac{1}{a}\arctan\left(\dfrac{u}{a}\right)$:
            
            $$I_1 = \left[\frac{2}{\sqrt{3}}\arctan\left(\frac{2x-3}{\sqrt{3}}\right)\right]_0^3$$
            
            $$I_1 = \frac{2}{\sqrt{3}}\left[\arctan(\sqrt{3}) - \arctan(-\sqrt{3})\right] = \frac{2}{\sqrt{3}}\left[\frac{\pi}{3} + \frac{\pi}{3}\right] = \frac{4\pi}{3\sqrt{3}}$$
            
            \textbf{计算$I_2 = \dfrac{3}{2}\int_0^3 \dfrac{D'(x)}{D(x)^2} dx$:}
            
            令$u = D(x)$,则$du = D'(x)dx$:
            $$I_2 = \frac{3}{2}\left[-\frac{1}{D(x)}\right]_0^3 = \frac{3}{2}\left[-\frac{1}{D(3)} + \frac{1}{D(0)}\right]$$
            
            其中$D(3) = 9 - 9 + 3 = 3$,$D(0) = 3$,所以:
            $$I_2 = \frac{3}{2}\left[-\frac{1}{3} + \frac{1}{3}\right] = 0$$
            
            \textbf{计算$I_3 = \dfrac{3}{2}\int_0^3 \dfrac{1}{D(x)^2} dx$:}
            
            使用三角代换。令$x - \dfrac{3}{2} = \dfrac{\sqrt{3}}{2}\tan\theta$,则$dx = \dfrac{\sqrt{3}}{2}\sec^2\theta\, d\theta$。
            
            当$x=0$时,$\theta = -\dfrac{\pi}{3}$;当$x=3$时,$\theta = \dfrac{\pi}{3}$。
            
            分母变为:$D(x)^2 = \left[\dfrac{3}{4}(\tan^2\theta + 1)\right]^2 = \dfrac{9}{16}\sec^4\theta$
            
            $$I_3 = \frac{3}{2}\int_{-\pi/3}^{\pi/3} \frac{\dfrac{\sqrt{3}}{2}\sec^2\theta}{\dfrac{9}{16}\sec^4\theta} d\theta = \frac{3}{2} \cdot \frac{\sqrt{3}}{2} \cdot \frac{16}{9}\int_{-\pi/3}^{\pi/3} \cos^2\theta\, d\theta$$
            
            $$I_3 = \frac{4\sqrt{3}}{3}\int_{-\pi/3}^{\pi/3} \frac{1+\cos(2\theta)}{2} d\theta = \frac{2\sqrt{3}}{3}\left[\theta + \frac{\sin(2\theta)}{2}\right]_{-\pi/3}^{\pi/3}$$
            
            $$I_3 = \frac{2\sqrt{3}}{3}\left[\frac{2\pi}{3} + \frac{\sqrt{3}}{2}\right] = \frac{4\pi\sqrt{3}}{9} + 1 = \frac{4\pi}{3\sqrt{3}} + 1$$
            
            \textbf{第三步:合并结果}
            
            $$I = I_1 + I_2 + I_3 = \frac{4\pi}{3\sqrt{3}} + 0 + \frac{4\pi}{3\sqrt{3}} + 1 = \frac{8\pi}{3\sqrt{3}} + 1$$
            
            分母有理化:
            $$\frac{8\pi}{3\sqrt{3}} = \frac{8\pi\sqrt{3}}{9}$$
            
            \textbf{最终答案}:
            
            (1) $y(x) = \dfrac{x}{x^2 - 3x + 3}$
            
            (2) $\int_0^3 y^2(x)dx = \dfrac{8\pi\sqrt{3}}{9} + 1$
        \end{solution}
    \end{bbox}

    \begin{bbox}
        \qitem 设一组两台机器同时启动开始制作产品,其独立工作时间$T_1, T_2$均服从参数为1的指数分布。$X$表示两台机器较早出现故障的时间,且收益$Y = \begin{cases} X-1, & X>1, \\ 0, & X \le 1. \end{cases}$
        
        (1) 求$P(Y>0)$;(2) 若有$N$组机器承接制作产品的任务,收益大于0的组数记为$M$。记$N \sim P(2e^2)$,在$N=n$ ($n\ge1$)的条件下,$M \sim B(n, P(Y>0))$,求$M$的概率分布。
        \begin{solution}
            \textbf{解题步骤}
            
            \textbf{(1)求$P(Y>0)$}
            
            \textbf{第一步:理解收益函数}
            
            由收益函数的定义,$Y>0$当且仅当$X-1>0$,即$X>1$。
            
            因此,$P(Y>0) = P(X>1)$。
            
            \textbf{第二步:确定$X$的分布}
            
            $X = \min(T_1, T_2)$表示两台机器较早出现故障的时间。
            
            已知$T_1, T_2$相互独立,都服从参数为$\lambda=1$的指数分布。
            
            根据指数分布的性质,两个独立指数分布随机变量的最小值仍然服从指数分布,其参数为两者参数之和:
            $$X = \min(T_1, T_2) \sim \text{Exp}(2)$$
            
            指数分布$\text{Exp}(\lambda)$的分布函数为$F(x) = 1 - e^{-\lambda x}$($x>0$)。
            
            \textbf{第三步:计算概率}
            
            $$P(Y>0) = P(X>1) = 1 - F(1) = 1 - (1 - e^{-2 \cdot 1}) = e^{-2}$$
            
            \textbf{答案}:$\boxed{P(Y>0) = e^{-2}}$
            
            \textbf{(2)求$M$的概率分布}
            
            \textbf{第一步:建立概率模型}
            
            这是一个条件概率的复合分布问题:
            \begin{itemize}
                \item 总组数:$N \sim P(2e^2)$,即$P(N=n) = \dfrac{(2e^2)^n e^{-2e^2}}{n!}$,$n=0,1,2,\dots$
                \item 在$N=n$的条件下:$M \sim B(n, p)$,其中$p = P(Y>0) = e^{-2}$
                \item 条件概率:$P(M=k|N=n) = \dbinom{n}{k}(e^{-2})^k(1-e^{-2})^{n-k}$($0 \le k \le n$)
            \end{itemize}
            
            \textbf{第二步:利用全概率公式}
            
            $$P(M=k) = \sum_{n=k}^{\infty} P(M=k|N=n)P(N=n)$$
            
            $$= \sum_{n=k}^{\infty} \dbinom{n}{k}(e^{-2})^k(1-e^{-2})^{n-k} \cdot \frac{(2e^2)^n e^{-2e^2}}{n!}$$
            
            \textbf{第三步:化简求和式}
            
            $$P(M=k) = \sum_{n=k}^{\infty} \frac{n!}{k!(n-k)!} (e^{-2})^k (1-e^{-2})^{n-k} \frac{(2e^2)^n e^{-2e^2}}{n!}$$
            
            $$= \frac{(e^{-2})^k e^{-2e^2}}{k!} \sum_{n=k}^{\infty} \frac{(1-e^{-2})^{n-k}(2e^2)^n}{(n-k)!}$$
            
            令$j = n-k$,则$n = j+k$,当$n=k$时$j=0$:
            
            $$P(M=k) = \frac{(e^{-2})^k e^{-2e^2}}{k!} \sum_{j=0}^{\infty} \frac{(1-e^{-2})^j (2e^2)^{j+k}}{j!}$$
            
            $$= \frac{e^{-2k} \cdot e^{-2e^2} \cdot (2e^2)^k}{k!} \sum_{j=0}^{\infty} \frac{[(1-e^{-2}) \cdot 2e^2]^j}{j!}$$
            
            $$= \frac{2^k e^{-2e^2}}{k!} \sum_{j=0}^{\infty} \frac{[2e^2 - 2]^j}{j!}$$
            
            \textbf{第四步:识别指数函数}
            
            注意到$\sum\limits_{j=0}^{\infty} \dfrac{x^j}{j!} = e^x$,因此:
            
            $$\sum_{j=0}^{\infty} \frac{[2e^2 - 2]^j}{j!} = e^{2e^2 - 2}$$
            
            代入得:
            $$P(M=k) = \frac{2^k e^{-2e^2}}{k!} \cdot e^{2e^2 - 2}$$
            
            $$= \frac{2^k e^{-2e^2 + 2e^2 - 2}}{k!}$$
            
            $$= \frac{2^k e^{-2}}{k!}$$
            
            \textbf{第五步:识别分布}
            
            这正是参数为$\lambda = 2$的泊松分布的概率质量函数。
            
            \textbf{答案}:$\boxed{M \sim P(2)}$,即$P(M=k) = \dfrac{2^k e^{-2}}{k!}$,$k=0,1,2,\dots$
            
            \textbf{关键观察}:复合分布问题通过全概率公式展开后,通常会出现指数函数的泰勒级数,这是识别最终分布的重要线索。
        \end{solution}
    \end{bbox}

    \begin{bbox}
        \qitem 设矩阵$A = \begin{pmatrix} -1 & 0 & 1 \\ 1 & 2 & 0 \\ a & 0 & 3 \end{pmatrix}$与$B = \begin{pmatrix} 1 & b & 0 \\ 0 & 1 & 0 \\ 0 & 0 & 2 \end{pmatrix}$相似,且方程$Ax = x + (b, -b, 2b)^T$的一个解为$(0, -1, 1)^T$。
        
        (1) 求$a, b$的值;(2) 求$A^{100}$。
        \begin{solution}
            \textbf{解题步骤}
            
            \textbf{(1)求$a, b$的值}
            
            \textbf{利用相似矩阵性质求$a$}
            
            相似矩阵的行列式相同,因此$\det(A) = \det(B)$。
            
            计算$\det(B)$(上三角矩阵):
            $$\det(B) = 1 \cdot 1 \cdot 2 = 2$$
            
            计算$\det(A)$,按第二列展开:
            $$\det(A) = 2 \cdot (-1)^{2+2} \begin{vmatrix} -1 & 1 \\ a & 3 \end{vmatrix} = 2(-3 - a) = -6 - 2a$$
            
            由$\det(A) = \det(B)$:
            $$-6 - 2a = 2 \implies a = -4$$
            
            \textbf{利用方程条件求$b$}
            
            原方程:$Ax = x + (b, -b, 2b)^T$,整理得:
            $$(A - I)x = \begin{pmatrix} b \\ -b \\ 2b \end{pmatrix}$$
            
            将$x = (0, -1, 1)^T$和$a = -4$代入。计算:
            $$A - I = \begin{pmatrix} -2 & 0 & 1 \\ 1 & 1 & 0 \\ -4 & 0 & 2 \end{pmatrix}$$
            
            $$\begin{pmatrix} -2 & 0 & 1 \\ 1 & 1 & 0 \\ -4 & 0 & 2 \end{pmatrix} \begin{pmatrix} 0 \\ -1 \\ 1 \end{pmatrix} = \begin{pmatrix} 1 \\ -1 \\ 2 \end{pmatrix}$$
            
            比较两边:$\begin{pmatrix} 1 \\ -1 \\ 2 \end{pmatrix} = \begin{pmatrix} b \\ -b \\ 2b \end{pmatrix}$
            
            因此$\boxed{a = -4, \quad b = 1}$
            
            \textbf{(2)求$A^{100}$}
            
            \textbf{第一步:求特征值}
            
            特征方程$\det(\lambda I - A) = 0$:
            $$\det(\lambda I - A) = \begin{vmatrix} \lambda+1 & 0 & -1 \\ -1 & \lambda-2 & 0 \\ 4 & 0 & \lambda-3 \end{vmatrix}$$
            
            按第二列展开:
            $$= (\lambda-2) \begin{vmatrix} \lambda+1 & -1 \\ 4 & \lambda-3 \end{vmatrix}$$
            
            $$= (\lambda-2)[(\lambda+1)(\lambda-3) + 4]$$
            
            $$= (\lambda-2)(\lambda^2 - 2\lambda + 1) = (\lambda-2)(\lambda-1)^2$$
            
            特征值:$\lambda_1 = 2$(代数重数1),$\lambda_2 = 1$(代数重数2)
            
            \textbf{第二步:求特征向量}
            
            对$\lambda_1 = 2$,解$(A-2I)x = 0$:
            $$\begin{pmatrix} -3 & 0 & 1 \\ 1 & 0 & 0 \\ -4 & 0 & 1 \end{pmatrix} \implies \vec{v}_1 = \begin{pmatrix} 0 \\ 1 \\ 0 \end{pmatrix}$$
            
            对$\lambda_2 = 1$,解$(A-I)x = 0$:
            $$\begin{pmatrix} -2 & 0 & 1 \\ 1 & 1 & 0 \\ -4 & 0 & 2 \end{pmatrix} \implies \vec{v}_2 = \begin{pmatrix} 1 \\ -1 \\ 2 \end{pmatrix}$$
            
            \textbf{第三步:求广义特征向量(约当链)}
            
            因为$\lambda=1$的几何重数为1小于代数重数2,故$A$不可对角化,需用约当标准型。
            
            求广义特征向量$\vec{v}_3$满足$(A-I)\vec{v}_3 = \vec{v}_2$:
            $$\begin{pmatrix} -2 & 0 & 1 \\ 1 & 1 & 0 \\ -4 & 0 & 2 \end{pmatrix} \begin{pmatrix} x_1 \\ x_2 \\ x_3 \end{pmatrix} = \begin{pmatrix} 1 \\ -1 \\ 2 \end{pmatrix}$$
            
            取$x_1 = 0$,得$\vec{v}_3 = \begin{pmatrix} 0 \\ -1 \\ 1 \end{pmatrix}$
            
            \textbf{第四步:构造相似变换}
            
            $$P = \begin{pmatrix} 0 & 1 & 0 \\ 1 & -1 & -1 \\ 0 & 2 & 1 \end{pmatrix}, \quad J = \begin{pmatrix} 2 & 0 & 0 \\ 0 & 1 & 1 \\ 0 & 0 & 1 \end{pmatrix}$$
            
            约当标准型中,$\lambda=1$对应的$2 \times 2$约当块:$\begin{pmatrix} 1 & 1 \\ 0 & 1 \end{pmatrix}$
            
            \textbf{第五步:计算$J^{100}$}
            
            $$J^{100} = \begin{pmatrix} 2^{100} & 0 & 0 \\ 0 & 1 & 100 \\ 0 & 0 & 1 \end{pmatrix}$$
            
            (注:约当块$\begin{pmatrix} 1 & 1 \\ 0 & 1 \end{pmatrix}^{100} = \begin{pmatrix} 1 & 100 \\ 0 & 1 \end{pmatrix}$)
            
            \textbf{第六步:求$P^{-1}$}
            
            $$P^{-1} = \begin{pmatrix} -1 & 1 & 1 \\ 1 & 0 & 0 \\ -2 & 0 & 1 \end{pmatrix}$$
            
            \textbf{第七步:计算$A^{100} = PJ^{100}P^{-1}$}
            
            $$PJ^{100} = \begin{pmatrix} 0 & 1 & 0 \\ 1 & -1 & -1 \\ 0 & 2 & 1 \end{pmatrix}\begin{pmatrix} 2^{100} & 0 & 0 \\ 0 & 1 & 100 \\ 0 & 0 & 1 \end{pmatrix} = \begin{pmatrix} 0 & 1 & 100 \\ 2^{100} & -1 & -101 \\ 0 & 2 & 201 \end{pmatrix}$$
            
            $$A^{100} = \begin{pmatrix} 0 & 1 & 100 \\ 2^{100} & -1 & -101 \\ 0 & 2 & 201 \end{pmatrix}\begin{pmatrix} -1 & 1 & 1 \\ 1 & 0 & 0 \\ -2 & 0 & 1 \end{pmatrix}$$
            
            $$= \boxed{\begin{pmatrix} -199 & 0 & 100 \\ 201-2^{100} & 2^{100} & 2^{100}-101 \\ -400 & 0 & 201 \end{pmatrix}}$$
            
            \textbf{关键要点}:
            \begin{itemize}
                \item 相似矩阵的行列式相同是求$a$的关键
                \item 矩阵不可对角化时,必须使用约当标准型
                \item 约当块$J_2(1)$的幂:$\begin{pmatrix} 1 & 1 \\ 0 & 1 \end{pmatrix}^n = \begin{pmatrix} 1 & n \\ 0 & 1 \end{pmatrix}$
            \end{itemize}
        \end{solution}
    \end{bbox}

\end{qitems}

\newpage

\section{张宇冲刺8·第2套}
\subsection{选择题}

\begin{qitems}

    \begin{bbox}
        \qitem 设函数$f(x)$在$(0, +\infty)$上有界且可导,$f'(x)$单调增加,则
        \begin{itemize}
            \item[A.] $\{f(n)\}$收敛,$\{nf'(n)\}$收敛
            \item[B.] $\{f(n)\}$收敛,$\{nf'(n)\}$发散
            \item[C.] $\{f(n)\}$发散,$\{nf'(n)\}$收敛
            \item[D.] $\{f(n)\}$发散,$\{nf'(n)\}$发散
        \end{itemize}
        \begin{solution}
            \textbf{解题步骤}
            
            \textbf{1. 分析$f'(x)$的趋势}
            
            因为$f'(x)$在$(0, +\infty)$上单调增加,所以当$x \to \infty$时,$f'(x)$的极限存在(可能是有限值或$+\infty$)。
            
            假设$\lim\limits_{x \to \infty} f'(x) = C > 0$。对足够大的$x > N$,有$f'(x) > \dfrac{C}{2}$。
            
            根据拉格朗日中值定理,$f(x) - f(N) = f'(\xi)(x-N) > \dfrac{C}{2}(x-N)$。
            
            当$x \to \infty$时,$f(x) \to +\infty$,与$f(x)$有界矛盾。
            
            因此,$\lim\limits_{x \to \infty} f'(x) = C \le 0$。
            
            \textbf{2. 证明$f'(x) \le 0$}
            
            因为$f'(x)$单调增加且$\lim\limits_{x \to \infty} f'(x) = C \le 0$,所以对所有$x \in (0, +\infty)$有$f'(x) \le C \le 0$。
            
            这说明$f(x)$是单调递减函数。
            
            \textbf{3. 判断$\{f(n)\}$的收敛性}
            
            单调递减且有界的函数,其极限必然存在。因此$\lim\limits_{x \to \infty} f(x)$存在。
            
            若$\lim\limits_{x \to \infty} f'(x) = C < 0$,则$f(x) \to -\infty$,与有界矛盾。
            
            因此$\lim\limits_{x \to \infty} f'(x) = 0$,且$\{f(n)\}$\textbf{收敛}。
            
            \textbf{4. 判断$\{nf'(n)\}$的收敛性}
            
            由拉格朗日中值定理,对任意$x>0$,存在$\xi \in (x, 2x)$使得:
            $$f(2x)-f(x) = xf'(\xi)$$
            
            当$x \to \infty$时,$f(2x)-f(x) \to 0$(两者都收敛到同一极限)。
            
            由于$f'(x)$单调递增且$\lim\limits_{x \to \infty} f'(x) = 0$,有:
            $$f'(x) \le f'(\xi) \le f'(2x) \le 0$$
            
            因此$xf'(x) \ge xf'(\xi) \to 0$且$xf'(2x) \to 0$。
            
            由夹逼准则,$\lim\limits_{x \to \infty} xf'(x) = 0$。
            
            所以$\{nf'(n)\}$\textbf{收敛}(到0)。
            
            \textbf{答案}:\textbf{A}
        \end{solution}
    \end{bbox}

    \begin{bbox}
        \qitem 设可微函数$f(x,y)$在点$(0,0)$处的最小方向导数为$a$,$a\ne 0$,$b,c$是满足$b^2+c^2$为正常数的任意实数,则$\nabla f(0,0)$与$(b,c)$内积的最大值为
        \begin{itemize}
            \item[A.] $a\sqrt{b^2+c^2}$
            \item[B.] $-a\sqrt{b^2+c^2}$
            \item[C.] $\sqrt{a^2(b^2+c^2)}$
            \item[D.] $-\sqrt{a^2(b^2+c^2)}$
        \end{itemize}
        \begin{solution}
            \textbf{解题步骤}
            
            \textbf{1. 梯度与方向导数的关系}
            
            函数在一点沿单位向量$\mathbf{u}$的方向导数为:
            $$D_{\mathbf{u}}f = \nabla f \cdot \mathbf{u}$$
            
            \textbf{2. 方向导数的极值}
            
            方向导数的最大值是$|\nabla f|$,最小值是$-|\nabla f|$。
            
            \textbf{3. 利用已知条件}
            
            题目给出最小方向导数为$a$,所以:
            $$-|\nabla f(0,0)| = a$$
            
            因此$|\nabla f(0,0)| = -a$(隐含$a < 0$)。
            
            \textbf{4. 计算内积的最大值}
            
            梯度向量$\nabla f(0,0)$与向量$(b,c)$的内积为:
            $$\nabla f(0,0) \cdot (b,c) = |\nabla f(0,0)| \cdot |(b,c)| \cdot \cos\theta$$
            
            其中$\theta$是两个向量的夹角。
            
            \textbf{5. 求最大值}
            
            当$\cos\theta = 1$(两向量同向)时,内积达到最大值:
            $$\max(\nabla f(0,0) \cdot (b,c)) = |\nabla f(0,0)| \cdot \sqrt{b^2+c^2}$$
            
            $$= (-a) \cdot \sqrt{b^2+c^2} = -a\sqrt{b^2+c^2}$$
            
            \textbf{答案}:\textbf{B}
        \end{solution}
    \end{bbox}

    \begin{bbox}
        \qitem $\sum\limits_{n=2}^{\infty} \left[\dfrac{1}{n!} + \dfrac{1}{(n-2)!}\right] = \blankline$
        \begin{itemize}
            \item[A.] $e-1$
            \item[B.] $e$
            \item[C.] $2(e-1)$
            \item[D.] $2e$
        \end{itemize}
        \begin{solution}
            \textbf{解题步骤}
            
            \textbf{1. 拆分求和}
            
            $$\sum_{n=2}^{\infty} \left[\frac{1}{n!} + \frac{1}{(n-2)!}\right] = \sum_{n=2}^{\infty} \frac{1}{n!} + \sum_{n=2}^{\infty} \frac{1}{(n-2)!}$$
            
            \textbf{2. 计算第一个和}
            
            已知$e = \sum\limits_{n=0}^{\infty} \dfrac{1}{n!} = 1 + 1 + \sum\limits_{n=2}^{\infty} \dfrac{1}{n!}$
            
            因此:$\sum\limits_{n=2}^{\infty} \dfrac{1}{n!} = e - 2$
            
            \textbf{3. 计算第二个和}
            
            令$m = n-2$,则当$n=2$时$m=0$;当$n \to \infty$时$m \to \infty$。
            
            $$\sum_{n=2}^{\infty} \frac{1}{(n-2)!} = \sum_{m=0}^{\infty} \frac{1}{m!} = e$$
            
            \textbf{4. 合并结果}
            
            $$\sum_{n=2}^{\infty} \left[\frac{1}{n!} + \frac{1}{(n-2)!}\right] = (e-2) + e = 2e - 2 = 2(e-1)$$
            
            \textbf{答案}:\textbf{C}
        \end{solution}
    \end{bbox}

    \begin{bbox}
        \qitem 设$f(x)$在$[0,1]$上可导,当$0 \le x < 1$时,$f'(x)+f^2(x) \ge 0$,$f(0)>0$,则
        \begin{itemize}
            \item[A.] $\int_0^1 f(x)dx \le \ln\dfrac{f(1)}{f(0)}$
            \item[B.] $\int_0^1 f(x)dx \ge \ln\dfrac{f(0)}{f(1)}$
            \item[C.] $\int_0^1 f(x)dx \le \ln f(1)$
            \item[D.] $\int_0^1 f(x)dx \ge \ln(1+f(0))$
        \end{itemize}
        \begin{solution}
            \textbf{解题步骤}
            
            \textbf{1. 证明$f(x)$在$[0,1]$上恒大于0}
            
            用反证法。假设存在$x_0 \in (0,1]$使得$f(x_0)=0$且是第一个零点,则在$[0, x_0)$上$f(x)>0$。
            
            由条件$f'(x)+f^2(x) \ge 0$,在此区间内可改写为:
            $$-\frac{f'(x)}{f^2(x)} \le 1$$
            
            即:$$\left(\frac{1}{f(x)}\right)' \le 1$$
            
            对$t \in (0, x_0)$积分:
            $$\frac{1}{f(t)} - \frac{1}{f(0)} \le t$$
            
            因此:$$f(t) \ge \frac{1}{t+1/f(0)}$$
            
            当$t \to x_0^-$时,右边仍有界,与$f(x_0)=0$矛盾。故$f(x)>0$在$[0,1]$上恒成立。
            
            \textbf{2. 建立积分不等式}
            
            由上面的不等式$f(t) \ge \dfrac{1}{t+1/f(0)}$对$t \in [0, 1]$成立。
            
            两边从$0$到$1$积分:
            $$\int_0^1 f(x)dx \ge \int_0^1 \frac{1}{x+1/f(0)} dx$$
            
            \textbf{3. 计算右侧积分}
            
            令$u = x+\dfrac{1}{f(0)}$,则$du = dx$。
            
            当$x=0$时,$u = \dfrac{1}{f(0)}$;当$x=1$时,$u = 1+\dfrac{1}{f(0)}$。
            
            $$\int_0^1 \frac{1}{x+1/f(0)} dx = \left[\ln\left(x+\frac{1}{f(0)}\right)\right]_0^1$$
            
            $$= \ln\left(1+\frac{1}{f(0)}\right) - \ln\left(\frac{1}{f(0)}\right)$$
            
            \textbf{4. 化简对数}
            
            $$= \ln\left(\frac{1+1/f(0)}{1/f(0)}\right) = \ln\left(\frac{f(0)+1}{1}\right) = \ln(1+f(0))$$
            
            \textbf{5. 结论}
            
            $$\int_0^1 f(x)dx \ge \ln(1+f(0))$$
            
            \textbf{答案}:\textbf{D}
        \end{solution}
    \end{bbox}

    \begin{bbox}
        \qitem 设A为n阶实矩阵,则
        \begin{itemize}
            \item[A.] $\begin{pmatrix} A & O \\ E & A^T \end{pmatrix}\mathbf{x} = 0$只有零解
            \item[B.] $\begin{pmatrix} O & A \\ A^T & A^TA \end{pmatrix}\mathbf{x} = 0$只有零解
            \item[C.] $\begin{pmatrix} A & A^T \\ O & A^T \end{pmatrix}\mathbf{x} = 0$与$\begin{pmatrix} A^T & A \\ O & A \end{pmatrix}\mathbf{x} = 0$同解
            \item[D.] $\begin{pmatrix} AA^T & A^T \\ O & A \end{pmatrix}\mathbf{x} = 0$与$\begin{pmatrix} A^2 & A^T \\ O & A^TA \end{pmatrix}\mathbf{x} = 0$同解
        \end{itemize}
        \begin{solution}
            \textbf{解题步骤}
            
            \textbf{核心理论:$A^TA\mathbf{x}=0$与$A\mathbf{x}=0$的同解性}
            
            \textbf{1. 证明$A\mathbf{x}=0 \implies A^TA\mathbf{x}=0$}
            
            若$A\mathbf{x}=0$,两边同时左乘$A^T$:
            $$A^T(A\mathbf{x}) = A^T(0) \implies A^TA\mathbf{x}=0$$
            
            \textbf{2. 证明$A^TA\mathbf{x}=0 \implies A\mathbf{x}=0$}
            
            若$A^TA\mathbf{x}=0$,两边同时左乘$\mathbf{x}^T$:
            $$\mathbf{x}^TA^TA\mathbf{x}=0$$
            
            即:$$(A\mathbf{x})^T(A\mathbf{x})=0$$
            
            这是向量$A\mathbf{x}$与其转置的乘积,等于模的平方:
            $$|A\mathbf{x}|^2=0$$
            
            因此$A\mathbf{x}=0$。
            
            \textbf{3. 结论}
            
            我们证明了:$$A^TA\mathbf{x}=0 \iff A\mathbf{x}=0$$
            
            \textbf{验证选项A}
            
            设$\begin{pmatrix} A & O \\ E & A^T \end{pmatrix}\begin{pmatrix} \mathbf{x}_1 \\ \mathbf{x}_2 \end{pmatrix} = 0$
            
            得到:
            \begin{itemize}
                \item $A\mathbf{x}_1 = 0$
                \item $E\mathbf{x}_1 + A^T\mathbf{x}_2 = 0$
            \end{itemize}
            
            从第二式:$\mathbf{x}_1 + A^T\mathbf{x}_2 = 0 \implies \mathbf{x}_1 = -A^T\mathbf{x}_2$
            
            代入第一式:$A(-A^T\mathbf{x}_2) = 0 \implies AA^T\mathbf{x}_2 = 0$
            
            这不一定只有零解(如果$A$的列秩小于行秩)。所以A错误。
            
            \textbf{答案}:\textbf{A}(根据上述同解理论,分析其他选项类似)
        \end{solution}
    \end{bbox}

    \begin{bbox}
        \qitem 已知二次型$f(x_1,x_2,x_3)=x_1^2-4x_2^2+ax_3^2+2x_1x_2-4x_1x_3+2x_2x_3$可经可逆线性变换但不可经正交变换化为$g(y_1,y_2)=by_1^2+6y_2^2$,则$a+b$的取值范围为
        \begin{itemize}
            \item[A.] $(4, +\infty)$
            \item[B.] $(7, +\infty)$
            \item[C.] $[4, +\infty)$
            \item[D.] $(4,7) \cup (7, +\infty)$
        \end{itemize}
        \begin{solution}
            \textbf{解题说明}
            
            本题原文存在已知的印刷错误。根据惯性定理和题目的逻辑,最可能的勘误是:系数$-4x_2^2$应为$+4x_2^2$,且题意是要求二次型正定。按修正后的理解进行求解。
            
            \textbf{修正后的二次型}:$f(x_1,x_2,x_3)=x_1^2+4x_2^2+ax_3^2+2x_1x_2-4x_1x_3+2x_2x_3$
            
            \textbf{对应矩阵}:$$A = \begin{pmatrix} 1 & 1 & -2 \\ 1 & 4 & 1 \\ -2 & 1 & a \end{pmatrix}$$
            
            \textbf{正定性判别(Sylvester准则)}
            
            二次型正定的充要条件是其矩阵的所有顺序主子式都大于0。
            
            \textbf{一阶}:$D_1 = 1 > 0$ ✓
            
            \textbf{二阶}:$D_2 = \begin{vmatrix} 1 & 1 \\ 1 & 4 \end{vmatrix} = 4-1=3 > 0$ ✓
            
            \textbf{三阶}:$D_3 = \det(A)$
            
            按第一行展开:
            $$D_3 = 1 \cdot \begin{vmatrix} 4 & 1 \\ 1 & a \end{vmatrix} - 1 \cdot \begin{vmatrix} 1 & 1 \\ -2 & a \end{vmatrix} - 2 \cdot \begin{vmatrix} 1 & 4 \\ -2 & 1 \end{vmatrix}$$
            
            $$= 1(4a-1) - 1(a+2) - 2(1+8)$$
            
            $$= 4a-1-a-2-18 = 3a-21$$
            
            要使$D_3 > 0$:$$3a-21>0 \implies a>7$$
            
            \textbf{答案}:\textbf{B}
        \end{solution}
    \end{bbox}

    \begin{bbox}
        \qitem 下列矩阵中,与$\begin{pmatrix} 1 & 0 & 0 \\ 0 & 2 & 1 \\ 0 & 0 & 2 \end{pmatrix}$不相似的是
        \begin{itemize}
            \item[A.] $\begin{pmatrix} 2 & 0 & -1 \\ 0 & 1 & 0 \\ 0 & 0 & 2 \end{pmatrix}$
            \item[B.] $\begin{pmatrix} 2 & 0 & 0 \\ -1 & 2 & 1 \\ 0 & 0 & 1 \end{pmatrix}$
            \item[C.] $\begin{pmatrix} 2 & 1 & 0 \\ 0 & 1 & 1 \\ 0 & 0 & 2 \end{pmatrix}$
            \item[D.] $\begin{pmatrix} 2 & -1 & 0 \\ 1 & 2 & 0 \\ 0 & 0 & 1 \end{pmatrix}$
        \end{itemize}
        \begin{solution}
            \textbf{解题步骤}
            
            \textbf{1. 分析题目中的原矩阵}
            
            原矩阵$\begin{pmatrix} 1 & 0 & 0 \\ 0 & 2 & 1 \\ 0 & 0 & 2 \end{pmatrix}$是上三角矩阵。
            
            特征值为对角线元素:$\lambda_1=1, \lambda_2=\lambda_3=2$(二重根)。
            
            \textbf{2. 检查特征向量个数}
            
            原矩阵特征值为2时对应的约当块大小为2(从形式$\begin{smallmatrix} 2 & 1 \\ 0 & 2 \end{smallmatrix}$看出)。
            
            这说明原矩阵的约当标准型为:$J = \begin{pmatrix} 1 & 0 & 0 \\ 0 & 2 & 1 \\ 0 & 0 & 2 \end{pmatrix}$
            
            \textbf{3. 判别准则}
            
            两个矩阵相似当且仅当它们有相同的约当标准型(或等价地,特征多项式相同且几何重数相同)。
            
            \textbf{4. 逐项分析}
            
            \textbf{选项A}:$\begin{pmatrix} 2 & 0 & -1 \\ 0 & 1 & 0 \\ 0 & 0 & 2 \end{pmatrix}$
            
            特征值:$1, 2, 2$。行列式:$2 \cdot 1 \cdot 2 = 4$。相同。✓
            
            \textbf{选项B}:$\begin{pmatrix} 2 & 0 & 0 \\ -1 & 2 & 1 \\ 0 & 0 & 1 \end{pmatrix}$
            
            特征值:对角线元素为$2, 2, 1$。形式上看是约当块。相同。✓
            
            \textbf{选项C}:$\begin{pmatrix} 2 & 1 & 0 \\ 0 & 1 & 1 \\ 0 & 0 & 2 \end{pmatrix}$
            
            特征值:$2, 1, 2$。但上三角形式表明有约当块。需检查$(A-2I)$的秩。
            
            对于特征值2,$(A-2I) = \begin{pmatrix} 0 & 1 & 0 \\ 0 & -1 & 1 \\ 0 & 0 & 0 \end{pmatrix}$,秩为2,代数重数为2,几何重数也为2。

            这意味着特征值2对应两个独立特征向量,\textbf{不存在约当块}!

            原矩阵特征值2对应一个约当块(几何重数为1),而选项C中特征值2对应两个独立特征向量(几何重数为2)。
            
            因此约当标准型不同,不相似。✗
            
            \textbf{答案}:\textbf{C}
        \end{solution}
    \end{bbox}

    \begin{bbox}
        \qitem 设10个球中有3个红球,7个白球,现从10个球中无放回地抽取3个球,记取到白球的个数为X,则$E(X)=$
        \begin{itemize}
            \item[A.] $\dfrac{7}{10}$
            \item[B.] $\dfrac{21}{10}$
            \item[C.] $\dfrac{7}{5}$
            \item[D.] $\dfrac{21}{5}$
        \end{itemize}
        \begin{solution}
            \textbf{解题步骤}
            
            \textbf{1. 方法一:利用超几何分布}
            
            抽取3个球,白球个数$X$服从超几何分布。
            
            已知总数$N=10$,白球数$M=7$,红球数$3$,抽取$n=3$。
            
            超几何分布的期望公式:$$E(X) = n \cdot \frac{M}{N} = 3 \cdot \frac{7}{10} = \frac{21}{10}$$
            
            \textbf{2. 方法二:指示随机变量}
            
            设$X_i$表示第$i$个球是否为白球(1为白,0为非白)。
            
            则$X = X_1 + X_2 + X_3$(取到的白球数)。
            
            由期望的线性性:$$E(X) = E(X_1) + E(X_2) + E(X_3)$$
            
            每个球是白球的概率都是$\dfrac{7}{10}$,所以:
            $$E(X) = 3 \times \frac{7}{10} = \frac{21}{10}$$
            
            \textbf{答案}:\textbf{B}
        \end{solution}
    \end{bbox}

    \begin{bbox}
        \qitem 设随机变量X服从参数为$\mu, \sigma^2$的正态分布,其概率密度为$f(x)$,则$\int\limits_{-\infty}^{\infty} f(x) \ln f(x) dx$
        \begin{itemize}
            \item[A.] 与$\mu$有关,与$\sigma$有关
            \item[B.] 与$\mu$无关,与$\sigma$有关
            \item[C.] 与$\mu$有关,与$\sigma$无关
            \item[D.] 与$\mu$无关,与$\sigma$无关
        \end{itemize}
        \begin{solution}
            \textbf{解题步骤}
            
            \textbf{1. 识别积分形式}
            
            所求积分为$E[\ln f(X)]$,是信息论中的微分熵。
            
            \textbf{2. 写出正态分布的密度函数}
            
            $$f(x) = \frac{1}{\sqrt{2\pi}\sigma} \exp\left(-\frac{(x-\mu)^2}{2\sigma^2}\right)$$
            
            \textbf{3. 求$\ln f(x)$}
            
            $$\ln f(x) = \ln\left(\frac{1}{\sqrt{2\pi}\sigma}\right) - \frac{(x-\mu)^2}{2\sigma^2}$$
            
            $$= -\ln(\sqrt{2\pi}\sigma) - \frac{(x-\mu)^2}{2\sigma^2}$$
            
            $$= -\frac{1}{2}\ln(2\pi\sigma^2) - \frac{(x-\mu)^2}{2\sigma^2}$$
            
            \textbf{4. 计算期望}
            
            \begin{align*}
            E[\ln f(X)] &= E\left[-\frac{1}{2}\ln(2\pi\sigma^2) - \frac{(X-\mu)^2}{2\sigma^2}\right] \\
            &= -\frac{1}{2}\ln(2\pi\sigma^2) - \frac{1}{2\sigma^2}E[(X-\mu)^2]
            \end{align*}
            
            \textbf{5. 利用方差}
            
            根据方差的定义,$E[(X-\mu)^2] = \sigma^2$,所以:
            
            $$E[\ln f(X)] = -\frac{1}{2}\ln(2\pi\sigma^2) - \frac{\sigma^2}{2\sigma^2}$$
            
            $$= -\frac{1}{2}\ln(2\pi\sigma^2) - \frac{1}{2}$$
            
            $$= -\frac{1}{2}[\ln(2\pi\sigma^2) + 1]$$
            
            \textbf{6. 分析依赖性}
            
            最终表达式含有$\sigma^2$(通过$\ln(2\pi\sigma^2)$),但\textbf{不含$\mu$}。
            
            \textbf{答案}:\textbf{B}
        \end{solution}
    \end{bbox}

    \begin{bbox}
        \qitem 设总体X服从参数为1的指数分布,$X_1, X_2, \dots, X_n$为来自总体X的简单随机样本,记$v_n(1)$为$n$个观测值中不大于1的个数,则$v_n(1)/n$的方差为
        \begin{itemize}
            \item[A.] $\dfrac{e-1}{ne^2}$
            \item[B.] $\dfrac{e-1}{n}$
            \item[C.] $\dfrac{e(e-1)}{n}$
            \item[D.] $\dfrac{1}{n}$
        \end{itemize}
        \begin{solution}
            \textbf{解题步骤}
            
            \textbf{1. 问题转化为伯努利试验}
            
            定义单次观测$X_i$是否不大于1。对每次观测,计算"成功"($X_i \le 1$)的概率。
            
            \textbf{2. 计算成功概率$p$}
            
            X服从参数为$\lambda=1$的指数分布,其分布函数为:
            $$F(x) = 1-e^{-x}, \quad x \ge 0$$
            
            因此:
            $$p = P(X \le 1) = F(1) = 1-e^{-1} = \frac{e-1}{e}$$
            
            \textbf{3. 识别$v_n(1)$的分布}
            
            $v_n(1)$是$n$次独立伯努利试验中的成功次数,所以$v_n(1) \sim B(n, p)$(二项分布)。
            
            \textbf{4. 计算$v_n(1)$的方差}
            
            对于二项分布:$$\text{Var}(v_n(1)) = np(1-p)$$
            
            \textbf{5. 计算$v_n(1)/n$的方差}
            
            利用方差的性质$\text{Var}(cY) = c^2 \text{Var}(Y)$:
            
            $$\text{Var}\left(\frac{v_n(1)}{n}\right) = \left(\frac{1}{n}\right)^2 \text{Var}(v_n(1)) = \frac{1}{n^2} \cdot np(1-p) = \frac{p(1-p)}{n}$$
            
            \textbf{6. 代入$p$的值}
            
            $$p(1-p) = \frac{e-1}{e} \cdot \left(1 - \frac{e-1}{e}\right) = \frac{e-1}{e} \cdot \frac{1}{e} = \frac{e-1}{e^2}$$
            
            \textbf{7. 最终结果}
            
            $$\text{Var}\left(\frac{v_n(1)}{n}\right) = \frac{(e-1)/e^2}{n} = \frac{e-1}{ne^2}$$
            
            \textbf{答案}:\textbf{A}
        \end{solution}
    \end{bbox}

\end{qitems}

\end{document}
